\section{Background}

\subsection{Purpose}

The $\psi(3770)$ decays are of particular interest because the resultant $D^0$
and $\xbar{D^0}$ are produced in a quantum correlated state. This allows for
better access to $D^0$ phase information introduced by the strong force. This in
turn is critical information in the accurate measurement of CP violation in
beauty decays (e.g. $B \rightarrow D K$), which is a large part of the
University of Bristol's LHCb programme. These $\psi(3770)$ decays can also be
used to measure symmetry violations in the \textbf{charm} system.

CP violation contributes to the matter anti-matter asymmetry that we see today,
i.e. why the universe is full of matter and not antimatter. At present the
amount of observed CP violation cannot account for the extent of
matter-antimatter asymmetry present. Analysis of the CP violation of both the
beauty and charm decays involved with the $\psi(3770)$ decays could both lead to
a better explanation of this matter-antimatter asymmetry, or to the discovery of
New Physics.

\subsection{ROOT Framework}

The \texttt{ROOT C++} framework and libraries were used for toy Monte Carlo (MC)
decays and to analyse the results find the invariant mass distributions and
decay angles in both the centre of mass frame and the laboratory frame. In
particular the
\href{http://root.cern.ch/root/html/TGenPhaseSpace.html\#TGenPhaseSpace:GetDecay}{\texttt{GetDecay()::TGenPhaseSpace}},
    \href{http://root.cern.ch/root/html/TLorentzVector.html\#TLorentzVector:M}{\texttt{M()::TLorentzVector}}
and
\href{http://root.cern.ch/root/html/TLorentzVector.html\#TLorentzVector:Angle}{\texttt{Angle()::TLorentzVector}}
methods were used to generate toy MC decays, get the invariant mass of a
4-vector and get the angle between two 4-vectors respectively.

% vim: set tw=80:
