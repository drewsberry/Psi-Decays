\section{Background}

\subsection{Purpose}
\begin{frame}{Purpose}
\begin{itemize}

    \Item $\psi(3770)$ decays are interesting because in a $\psi(3770)$ decay,
        the $D^0$ and $\xbar{D^0}$ are produced in a quantum correlated state.
    
    \Item This quantum correlation allows better access to $D^0$ strong phase
        information (the phases introduced by the strong force). $D^0$ phase
        information is critical information that allows us to accurately measure
        CP violation in beauty decays ($B \rightarrow D~K$, a big part of
        Bristol's LHCb programme).
    
    \Item The amount of CP violation that occurs in such processes is important
        because CP violation helps to explain matter-antimatter asymmetry (i.e.
        why is the universe full of matter and not anti-matter?).

\end{itemize}
\end{frame}

\begin{frame}{Purpose}
\begin{itemize}

    \Item \textbf{But} at present the amount of CP violation measured to occur does
        not account for the asymmetry that we see; there simply is not enough CP
        violation to justify the amount of asymmetry we see.
    
    \Item Interestingly, $\psi(3770)$ decays can also be used to measure symmetry
        violations in the \textbf{charm} system, thanks to the quantum correlations
        in the decay.
    
    \Item Analysis of the amount of CP violation in both beauty and charm could lead
        to a better explanation of matter-antimatter asymmetry, or to the discovery
        of \textit{New Physics}\texttrademark.

\end{itemize}
\end{frame}

\subsection{ROOT}
\begin{frame}[fragile]{ROOT}
    \begin{itemize}

    \Item The \texttt{ROOT C++} libraries were used to simulate and analyse the
        decay events. 
    
    \Item In particular the
        \href{http://root.cern.ch/root/html/TGenPhaseSpace.html}{\texttt{TGenPhaseSpace}}
        class was used to generate Monte Carlo (MC) phase space for n-body
        decays of constant cross-section as per Frederick James' 1968 paper,
        \href{http://cds.cern.ch/record/275743}{\textit{Monte Carlo Phase
        Space}}.
    
    \Item The \texttt{GetDecay()} method, inheriting from
        \texttt{TGenPhaseSpace}, generates Lorentz vectors for the decay
        products given.

\end{itemize}
\end{frame}

\begin{frame}[fragile]{Code}
\begin{minted}[linenos]{c++}

TLorentzVector parent(0.0,0.0,0.0, PARENTMASS);
Double_t masses[2] = {DAUGHTER1MASS, DAUGHTER2MASS};

TGenPhaseSpace event;
event.SetDecay(parent, 2, masses);

TLorentzVector *daughter1 = event.GetDecay(0);
TLorentzVector *daughter2 = event.GetDecay(1);

\end{minted}
\end{frame}

\begin{frame}{Code}

This example code demonstrates the basic method; once you start looking at
longer decay chains and relativistically boosting and analyzing them, it gets a
lot more complicated! (And uglier...)

\end{frame}

